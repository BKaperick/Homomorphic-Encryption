\documentclass[a4paper,11pt, oneside]{article}
\usepackage{C:/Programs/Vim/TiaStyle}
\input{C:/Programs/Vim/style.tex}

\newcommand{D}{\mathcal{D}_{\gamma,\rho}(p)}

\title{Analysis of Somewhat Homomorphic Encryption Over the Integer Ring}
\author{Bryan Kaperick}

\begin{document}
\maketitle\newpage
\section{Preliminaries}

Traditionally, the modulus operator can be defined as follows
\begin{definition}
  Define $q_a(b) = \floor{\frac{b}{a}}$.  Then, define $a\Mod{b} = b - q_a(b)a$, which is equivalent to setting $a\Mod{b}$ to be the representative in $[0,b)$ for the residue class containing $a$ for the congruence relation of congruence modulo $b$.
\end{definition}

However, for the purposes of this paper, it will be seen that a slightly altered definition is much more convenient.
\begin{definition}
  Define $q_a(b) = \round{\frac{a}{b}}$, where $\round{\cdot}$ returns the nearest integer to the input value (rounding up for multiples of one-half).  Again, define $a\Mod{b} = b - q_a(b)a$.
\end{definition}

In the original definition, it is true that $a\Mod{b}

\section{Goals of Scheme}
\section{Motivation for Approach}
\section{Implementation}
    \subsection{\textit{Special} Distribution, \D}
    \subsection{Proof of Validity}
\section{Attacks}
\subsection{Least Significant Bit Guessing}
\subsection{Solving Approximate GCD}



\end{document}

